% !TeX encoding = UTF-8
\documentclass[type=bachelor,printmode=twoside,]{cquthesis}
% 选项:
% type=[bachelor|master|doctor],			% 毕业论文类型,必选
%	printmode=[oneside|twoside|auto],		% 可选,一般默认采用auto按页数要求自动判定
%	liberalformat,											% 本科生可选,使用文学类论文标题格式
% continuoustoc,											%	目录和索引不会自动新开一页,适用于索引内容较少时
% draft,															%	草稿模式,不加载图片,加快预览速度

% 请在cquthesis.sty文件中定义其他会用到的宏包和自己的变量
\usepackage{cquthesis}

% 定义所有的图片文件在 figures 子目录下
\graphicspath{{figures/}}

%*** 使用这个命令只渲染你想查看的部分,提升工作效率
%\includeonly{contents/experiment,contents/analysis,}

\begin{document}

\cqusetup{
%	************	注意	************
%	* 1. \cqusetup{}中不能出现全空的行,如果需要全空行请在行首注释
%	* 2. 不需要的配置信息可以放心地坐视不理、留空、删除或注释(都不会有影响)
%	*
%	********************************
% ===================
%	论文的中英文题目
% ===================
  ctitle = {\ce{C60}表面的纳米形貌的可控性转变\\兼论\LaTeX{}在论文排版中的运用},
  etitle = {To Use \LaTeX{} in the Typeseting of\\Graduating Work for CQU},
% ===================
% 作者部分的信息
% ===================
  cauthor = 李振楠,	% 你的姓名,以下每项都以英文逗号结束
  eauthor = Zhennan~Li,	% 姓名拼音,~代表不会断行的空格
  studentid = 20128888,	% 仅本科生,学号
  csupervisor = 孙麟~~教授,	% 导师的姓名
  esupervisor = {Prof.~Lin Sun},	% 导师的姓名拼音
  cassistsupervisor = {}, % 本科生可选,助理指导教师姓名,不用时请留空为{}
  cextrasupervisor = {}, % 本科生可选,校外指导教师姓名,不用时请留空为{}
  eassistsupervisor = {}, % 本科生可选,助理指导教师或/和校外指导教师姓名拼音,不用时请留空为{}
  cpsupervisor = 丁小明~~工程师, % 仅专硕,兼职导师姓名
  epsupervisor = Eng.~Xiaoming~Ding,	% 仅专硕,兼职导师姓名拼音
  cclass = 工学,	% 博士生和学硕填学科门类,学硕填学科类型
  edgree = {Degree of Master of Enginnering},	% 专硕填Professional Degree,其他按实情填写
% 提示:如果内容太长,可以用\zihao{}命令控制字号,作用范围:{}内
  cmajor = {\zihao{-4}材料科学与工程(材料科学专业方向)},	% 专硕不需填,填写专业名称
  emajor = Material Science and Engineering, % % 专硕不需填,填写专业英文名称
% ===================
% 底部的学院名称和日期
% ===================
  cdepartment = 材料科学与工程学院,	%学院名称
  edepartment = College of Material Science and Engineering,	%学院英文名称
% ===================
% 封面的日期可以自动生成(注释掉时),也可以解除注释手动指定,例如:二〇一六年五月
% ===================
%	mycdate = {中文日期},
%	myedate = {Date in English},
}% End of \cqusetup
% ===================
%
% 论文的摘要
%
% ===================
\begin{cabstract}	% 中文摘要
	本文档是\cquthesis{}的示例文档。
		
	\cquthesis{}是重庆大学毕业论文的\LaTeX{}模板,支持学士、硕士、博士论文的排版。合理使用本模板可以大大减轻重庆大学毕业生在毕业论文撰写过程中的排版工作量。
	
	\cquthesis{}根据重庆大学《重庆大学本科设计(论文)撰写规范化要求(2007年修订版)》和《重庆大学博士、硕士论文撰写格式标准(2007年修订版)》编写,力求合规,简洁,易于实现,用户友好。
	
	本模板的特色:
  \begin{itemize}
  	\item 支持重庆大学本科(文学、理工)、硕士(学术、专业)、博士的毕业论文格式;
  	\item 内置封面、目录、索引、授权书等论文部件,可按需自动生成;
  	\item 自动侦测文档页数,生成相应的单面打印/双面打印PDF文件;
  	\item 预置一批优化过的宏包和小功能,包含国际标准单位、化学式支持、三线表等,可按需开启。
  \end{itemize}
\end{cabstract}
% 中文关键词,请使用英文逗号分隔:
\ckeywords{重庆大学,\LaTeX,\LaTeXe,论文,模板}

\begin{eabstract}	% 英文摘要
	LaTeX is a document preparation system for high-quality typesetting. It is most often used for medium-to-large technical or scientific documents but it can be used for almost any form of publishing.
	
  LaTeX contains features for:
\begin{enumerate}
  	\item Typesetting journal articles, technical reports, books, and slide presentations.
  	\item Control over large documents containing sectioning, cross-references, tables and figures.
  	\item Typesetting of complex mathematical formulas.
  	\item Advanced typesetting of mathematics with AMS-LaTeX.
  	\item Automatic generation of bibliographies and indexes.
  	\item Multi-lingual typesetting.
  	\item Inclusion of artwork, and process or spot colour.
  	\item Using PostScript or Metafont fonts.
  \end{enumerate}
  (Quote from \textit{https://latex-project.org/intro.html})  
\end{eabstract}
% 英文关键词,请使用英文逗号分隔,关键词内可以空格:
\ekeywords{bachelor, master, doctor, all support, white space is okay here}

% 封面和摘要配置完成

%%% 封面部分
\makecover

\frontmatter
%% 摘要
\makeabstract
%% 目录
\tableofcontents
%% 插图索引
\listoffigures
%% 表格索引
\listoftables
%% 公式索引
\listofequations
%% 符号对照表
% !TeX encoding = UTF-8
% 环境用两个长度参数,分别定义左边距以及词条和解释的水平距离,可自己调试以达美观(全去掉时默认:20mm,30mm)
\begin{denotation}[10mm][40mm]
	\item[CQU] 重庆大学(Chongqing University)的英文缩写
	\item[\LaTeX] 一个很棒的排版系统
	\item[\LaTeXe] 一个很棒的排版系统的最新稳定版
	\item[\XeTeX] \LaTeX{}的好兄弟,事实上他有很多个兄弟,但是这个兄弟对各种语言的支持能力都很强
	\item[CTeX宏集] 成套的中文\LaTeX{}解决方案,由一帮天才们开发
	\item[\ce{H2SO4}] 硫酸
	\item[$ e^{\pi{}i}+1=0$] 一个集自然界五大常数一体的炫酷方程
	\item[\ce{2H2 + O2 -> 2H2O}] 一个昂贵的生成生命之源的方程式
\end{denotation}

\endinput



%%% 主体部分(绪论开始,结论为止)
%* 子文件的多少和内容由你决定,基本原则是提速预览、脉络清晰、管理容易。
\mainmatter

\chapter{绪论}
\section{欢迎!}
欢迎来到\cquthesis{}示例文档!

本文档使用\cquthesis{}本身作为模板,即\pkg{cquthesis.cls}, \pkg{cquthesis.sty}和\pkg{cquthesis.cfg},旨在展现\cquthesis{}的使用方法。请结合\cquthesis{}用户手册和本文档源代码进行学习和操作。

祝毕设成功,答辩拿优!Happy Texing!

本文档编译时使用的\cquthesis{}版本为\version{}。

请留意到本链接检查更新:\url{https://github.com/nanmu42/CQUThesis}

\section{关于\LaTeX{}}
\noindent{\heiti{}提示:}{\kaishu{}下面是一些基本思路和知识,如果你已经对\LaTeX{}比较熟悉,请直接跳转到第\ref{txt:FreqCmd}节。}

\subsection{关于推荐重庆大学开设\TeX 相关课程并推广其运用的提议}
这一小节是对\href{http://jq.qq.com/?_wv=1027&k=2HvYu95}{重庆大学\TeX 用户组}所撰写的提案的简介。

本提案从介绍排版系统\TeX 的背景和特点开始,从研究生期刊论文投稿以及毕业生毕业论文排版工作这两个维度阐述了引入\TeX 作为一种与Office Word平行的写作系统的优势和必要性,最终提出一套基于我校重庆大学实际情况,有效可行的实施方案。

这份提案可以作为新手从全局认识\TeX 的入门材料,提案的下载地址是:\url{https://github.com/CQUtug/TeXProposal}

\subsection{\LaTeX{}小传}
\LaTeX{}是\TeX{}的改进版本,后者由Knuth(高德纳)在上世纪七十年代研发,包含\TeX{}排版程序和Plain \TeX{}宏集这两部分。Plain \TeX{}可以看做是一种既定语法的编程语言,源代码对应文件后缀为\pkg{.tex},而\TeX{}程序对源代码进行解析,编译,得到排版结果。上世纪八十年代,\LaTeX{}对Plain \TeX{}的语言体系进行了升级和重构,使得\TeX{}的易用性获得了质的提升。

\TeX{}有着很多分支,比如\LaTeX{}, \LuaTeX{}和\XeTeX{}。每个分支的产生都是为了解决不同的问题。其中,\XeTeX{}提供了对东亚字体(中日韩)的原生支持。


\subsection{\LaTeX{}背后的思路}

\LaTeX{}的思路也许你已经有所耳闻,即{\heiti{}内容和样式分离。}它有些像HTML编辑器,遵循\textsf{WYTIWYG}原则\footnote{What you think is what you get. 所想即所得。}。这也是它和Word这一类遵循\qthis{所见即所得}\footnote{\textsf{WYSIWYG} -- What you see is what you get.}原则的文档编辑器的最大不同。

在你使用\LaTeX{}写作的时候,大部分时间你只需要关心内容本身,而\LaTeX{}就像你的编辑,按照样式要求(模板和宏包)为你排版。


\section{\cquthesis{}背后的思路}

\cquthesis{}秉承\LaTeX{}的思路,旨在为你解决论文内容以外的大部分问题。

出于性能和管理方面的考虑,\cquthesis{}使用分布式的源文件方案,将论文的各个部分(通常以章为单位)分散到tex文件中,然后在主文档\pkg{main.tex}中统一处理。\figref{fig:filetree}展示了一个可能的文件目录情况。
\begin{figure}[htb]
	\dirtree{%
		.1 \myfolder{pink}{工作文件夹}.
		.2 \myfolder{cyan}{\pkg{cquthesis.cls}}.
		.2 \myfolder{cyan}{\pkg{cquthesis.cfg}}.
		.2 \myfolder{cyan}{\pkg{cquthesis.sty}}.
		.2 \myfolder{cyan}{main.tex}.
		.2 \myfolder{cyan}{contents}.
		.3 \myfolder{lime}{introduction.tex}.
		.3 \myfolder{lime}{experiment.tex}.
		.3 \myfolder{lime}{analysis.tex}.
		.3 \myfolder{lime}{conclusion.tex}.
		.2 \myfolder{cyan}{figures}.
		.3 \myfolder{lime}{myCat.png}.
		.3 \myfolder{lime}{dogEatsSandwiches.jpg}.
		.2 \myfolder{cyan}{ref}.
		.3 \myfolder{lime}{refs.bib}.
	}%\dirtree
\caption[\cquthesis{}文件结构图示]{\cquthesis{}文件结构图示,出于测试的原因,这个标题被故意填充得很长,这里,你可以结合本文文档代码看到插图索引中是如何处理这个问题的。}
\label{fig:filetree}
\end{figure}







\chapter{实验参数和流程}
\section{5分钟语法参考}

{\kaishu 要流畅使用\cquthesis 需要用户对\LaTeX 以及\textsc{Bib}\TeX 有一定了解,下面这个语法参考只能起到抛砖引玉的作用。如果你从来没有接触过\LaTeX 或者\XeTeX ,建议先学习相关知识,磨刀不误砍柴工。}
\medskip
\begin{itemize}
	\item \LaTeX 源文件中,主要有三种元素:你的文字,命令,以及环境;
	\item 直接输入即可你想要写的文字即可,对于英文,文字间多于一个的空格都会转为一个空格;
	\item 如果你想开启一个新的自然段,请在写新内容前空一个(或多个)全空的行;
	\item \LaTeX 的命令{\heiti 全部}都以\cs{ }开头,例如\cs{XeTeX}可以生成\XeTeX ;
	\item 有的命令{\heiti 必须}带参数,比如\cs{zihao\{-4\}}可以将命令之后的内容的字号调整为小四;
	\item 有的命令能带可选参数,例如\cs{usingpackage\{metalogo\}}可以载入\pkg{metalogo}宏包;
	\item 宏包中有宏包作者自己定义的命令,能够让你更容易地完成某些事情,比如\pkg{mhchem}能够引入让你方便地表示化学式的命令\cs{ce};
	\item \LaTeX 的源代码主要分为两个部分,导言部分和文档部分。其中,文档部分以\cs{begin\{document\}}开头,以\cs{end\{document\}}结尾,只有在这个范围内你才能完成排版工作;
	\item \LaTeX 对(简单或复杂的)数学式的支持是其一大亮点,数学环境使用\texttt{\${ }\$}包裹;
	\item 环境由\csgo{begin}{环境名}开头,以\csgo{end}{环境名}结尾,是的,文档部分是一个巨大的环境;
	\item 报错说没有这个命令?检查是否载入了必要的宏包,再检查命令后面是否直接跟随了汉字,在它们之间加个空格就好;
	\item \LaTeX 是一门语言,新手经常会遇到无法编译通过的语法错误,这时建议你仔细检查花括号是否平衡,命令是否敲错,参数数目和类型是否正确,如果还是不行,可以在网络上搜索一番或者问问同事。
	\item 命令之间或者之内的空格和缩进以及回车不是必须的,事实上没有它们\LaTeX 也可以正常工作,但是代码的可读性就会大打折扣了;
	\item 对了,使用\texttt{\%}来开启一个行注释,注释的内容不参与编译,你可以在这里写下自己的小秘密;
	\item 有质量的国内\TeX 社区是\textsc{CTeX}社区,更有质量的国外的是\textsf{StackExchange};
	\item \TeX\textsc{Studio}是一个很棒的\LaTeX 编辑环境,推荐你尝试一番。
\end{itemize}

\section{查询文档}
在你对宏包或者环境包有疑问的时候,可以再命令行中输入:

\texttt{texdoc 宏包或环境名称}

回车后对应的用户文档会自动打开。






\chapter{实验结果和分析}
\section{实现范例}
\section{注意事项和提示}
\chapter{结论}
\chapter{结论2}
\chapter{结论3}
\chapter{结论4}


%%% 后置部分(致谢开始)
\backmatter

%% 致谢
\chapter{致\hskip2\ccwd{}谢}	% 致谢之间空两个汉字
这个模板是站在巨人肩膀上的成果,感谢\LaTeXe{}计划,感谢CTeX开发组提供的中文解决方案,感谢薛瑞尼副教授(Github: xueruini/ThuThesis),感谢WeiJianWen同学(Github: weijianwen/SJTUThesis),感谢中国科学技术大学TeX用户组(Github: ustctug/gbt-7714-2015)。向你们致以真诚的问候和感激。

%% 参考文献
% 注意:至少需要引用一篇参考文献,否则下面两行会引起编译错误。
%\bibliographystyle{cqunumerical.bst}		%另有cquauthoryear供硕士博士生选用
%\bibliography{ref/refs}


%% 附录(以第A章开始,证明、推导、程序、个人简历等)
\appendix

%% 个人简历
%\include{contents/resume}
%% 原创声明和授权说明书,可选:用扫描页替换
%\cquauthpage[contents/authscan.pdf]
\cquauthpage

\end{document}
